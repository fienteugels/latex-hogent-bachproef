%---------- Inleiding ---------------------------------------------------------

\section{Inleiding}%
\label{sec:inleiding}

Moderne webapplicaties worden meestal ontwikkeld met een sterke focus op de technische functionaliteiten. Hierdoor krijgt gebruiksvriendlijkheid pas laat in het ontwikkelingsproces aandacht, wat leidt tot fouten, frustraties en soms zelfs tot het afhaken van eindgebruikers.
Dit probleem is duidelijk zichtbaar in de concrete casus van dit onderzoek: een React-gebaseerde webapplicatie waarin gebruikers moeite hebben met het doorlopen van een meerstapsformulier.

De doelgroep van dit onderzoek zijn IT-professionals, meer specifiek full-stack developers en software engineers die webapplicaties bouwen met moderne frameworks zoals React.
Zij zijn verantwoordlijk voor de technische implementatie en de gebruikerservaring. Ze hebben baat bij inzichten over wanneer usability-testing het meest effectief is.

De probleemstelling luidt dat gebruikers problemen ervaren bij het doorlopen van een meerstapsformulier in een React-applicatie, en dat er geen systematische manier bestaat om gebruikersgedrag vroegtijdig in het ontwikkelingsproces te meten. Hierdoor worden usability-problemen pas laat in het ontwikkelingsproces zichtbaar, wanneer aanpasisngen vaak duur en complex zijn.

De centrale onderzoekvraag is daarom: Wat is het effect van usability‑tests die vroeg, laat of zowel vroeg als laat in het ontwikkelingsproces worden uitgevoerd op de gebruiksvriendelijkheid van een React‑applicatie, gemeten via Application Insights‑data, taakvoltooiingstijd, fouten en SUS‑scores?

De onderzoeksdoelstelling is om te bepalen wanneer usability-testing de grootste impact he\\-eft op de gebruiksvriendelijkheid van een webapplicatie. Het eindresultaat van deze bachelorproef bestaat ut:
{\begin{itemize}
  \item een proof-of-concept React-applicatie met geïntegreerde Application Insights
  \item een datagedreven vergelijking van drie testmomenten
  \item aanbevelingen voor developers over wanneer usablity-testing het meest waardevol is
\end{itemize}}
%---------- Stand van zaken ---------------------------------------------------

\section{Literatuurstudie}%
\label{sec:literatuurstudie}

Gebruiksvriendelijkheid is een essentieel onderdeel van een kwalitatieve webapplicatie. Volgens ~\autocite{Nielsen2021} bepaalt usability hoe efficiënt, foutloos en tevreden gebruikers taken kunnen uitvoeren. ~\autocite{Norman2013} benadrukt dat kleine ontwerpbeslissingen grote gevolgen kunnen hebben voor gebruikersfouten en frustratie. 
~\autocite{Krug2014} stelt dat gebruikers intuïtief moeten kunnen navigeren zonder cognitieve belasting, wat vooral belangrijk is bij meerstapsformulieren.

~\autocite{SauroLewis2016} beschrijven hoe usability objectief meetbaar kan worden gemaakt via taakvoltooiingstijd, fouten succesratio's en tevredenheidsscores zoals de System Usability Scale (SUS). Deze meetmethoden vormen de basis voor het kwantitatie evalueren van de casusapplicatie.

In het oplossingsdomein tonen ~\autocite{RubinChisnell2008} aan dat usability-tests met kleine groepen gebruikers al waardevolle inzichten opleveren. ~\autocite{GothelfSeiden2016} pleiten voor Lean UX, waarbij testen vroeg in het proces cruciaal zijn om iteratief te verbeteren.~\autocite{Rosenfeld2015} benadrukken het belang van duidelijke information architecture voor navigatie en formulierflows.

Daarnaast tonen Microsoft-documentaties aan dat Application Insights een krachtuge tool is om gebruikersgedrag automatisch te meten via page views, click events, errors, funnels en time-per)step. Deze datagegevens aanpak maakt het moeglijk om usability-problemen objectief te detecteren en te vergelijken tussen verschillende testmomenten.

% Voor literatuurverwijzingen zijn er twee belangrijke commando's:
% \autocite{KEY} => (Auteur, jaartal) Gebruik dit als de naam van de auteur
%   geen onderdeel is van de zin.
% \textcite{KEY} => Auteur (jaartal)  Gebruik dit als de auteursnaam wel een
%   functie heeft in de zin (bv. ``Uit onderzoek door Doll & Hill (1954) bleek
%   ...'')


%---------- Methodologie ------------------------------------------------------
\section{Methodologie}%
\label{sec:methodologie}

Het onderzoek bestaat uit twee grote componenten: een technisch component en een onderzoekend component.

\subsection{Technisch component (Proof of Concept)}
De technische basus van dit onderzoek is een React-applicatie met een meerstapsformulier. In deze applicatie wordt Microsoft Application Insights geïntegreerd om automatisch gebruikersinteracties te registreren. Deze tool verzameld onder andere:
\begin{itemize}
  \item page views
  \item click eventsform interactions
  \item validatiefouten
  \item time-per-step
  \item funnels en drop-off-punten
\end{itemize}
Deze data vormt de objectieve basis voor het vergelijken van de drie testmomenten.

\subsection{Onderoekend component}
Het onderzoek vergelijkt drie testmomenten in het ontwikkelingsproces:
\subsubsection{Vroege usability-test (begin van de ontwikkeling)}
\begin{itemize}
  \item Kleine groep gebruikers (5-10 personen)
  \item Taken: het volledige formulier doorlopen
  \item Metingen: taakvoltooiingstijd, fouten, SUS-score, Insights-data
  \item Doel: vroeg problemen detecteren en documenteren
\end{itemize}

\subsubsection{Late usability-test (einde van de ontwikkeling)}
\begin{itemize}
  \item Zelfde taken en metingen
  \item Doel: evalueren of de aangebrachte verbeteringen effectief zijn
\end{itemize}

\subsubsection{Begin én einde (combinatie)}
\begin{itemize}
  \item Vergelijking van beide testmomenten
  \item Analyse van verschillen in
  \begin{itemize}
    \item fouten
    \item tijd
    \item drop-off
    \item tevredenheidsscoresInsights-metrics
  \end{itemize}
\end{itemize}

\subsection{Onderzoekstechnieken}
\begin{itemize}
  \item Literatuurstudie om het theoretisch kader te bepalen
  \item Experimentele casestudy met drie testmomenten
  \item Dataverzameling via Application Insights
  \item Vergelijkende analyse van de drie testmomenten
  \item Proof-of-concept als technische onderbouwingen
\end{itemize}

\subsection{Tijdsplanning}
\begin{itemize}
  \item \textbf{December 2025 -- Januari 2026:} Opzetten van de React-applicatie, integratie van Application Insights en verwerken van feedback op het onderzoeksvoorstel.
  \item \textbf{Februari 2026:} Uitvoeren van de vroege usabi\\-lity-test (begin van het ontwikkelingsproces) en doorvoeren van eerste verbeteringen.
  \item \textbf{Maart 2026:} Verwerken van literatuur en methodologie voor de draft-indiening en bespreken met de promotor.
  \item \textbf{April 2026:} Uitvoeren van de late usability-test (einde van het ontwikkelingsproces) en vergelijken met de vroege testresultaten.
  \item \textbf{April -- Mei 2026:} Analyseren van alle data en uitwerken van resultaten, conclusies en aanbevelingen.
  \item \textbf{Mei 2026:} Finaliseren en indienen van de bachelorproef.
\end{itemize}

%---------- Verwachte resultaten ----------------------------------------------
\section{Verwacht resultaat, conclusie}%
\label{sec:verwachte_resultaten}
\subsection{Verwacht resultaat}
Op basis van de literatuur en de aard van het onderzoek wordt er verwacht dat:
\begin{itemize}
  \item Vroege usability-tests meer problemen blootleggen dan late tests
  \item Late tests vooral bevestigen of verbeteringen effectief zijn
  \item De combinatie van begin én einde de meest volledige inzichten oplevert
  \item Application Insight duidelijke patronnen zal vaststellen in:
  \begin{itemize}
    \item fouten
    \item drop-off
    \item time-per-step
    \item navigatieproblemen
  \end{itemize}
\end{itemize}
Voor de doelgroep, namelijk full-stack developers, levert dit onderzoek concrete aanbevelingen over wanneer usability-testing het meeste waarde heeft.
\subsection{Verwachte conclusie}
Het is waarschijnlijk dat usability-tests die vroeg in het ontwikkelingsproces worden uitgevoerd een grote impact hebben op de uiteindelijke gebruiksvriendelijkheid dan tests die pas op het einde plaatsvinden. Door op beide momenten te testen zal de meest effectieve aanpak ontstaan om webapplicaties gebruikevriendelijker te maken.